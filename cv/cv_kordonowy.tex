\documentclass[11pt,a4paper,sans]{moderncv}        % possible options include font size ('10pt', '11pt' and '12pt'), paper size ('a4paper', 'letterpaper', 'a5paper', 'legalpaper', 'executivepaper' and 'landscape') and font family ('sans' and 'roman')'
% \usepackage{fontawesome}
% \usepackage{umvs}
\usepackage{lmodern}

% moderncv themes
\moderncvstyle{classic}                             % style options are 'casual' (default), 'classic', 'banking', 'oldstyle' and 'fancy'
\moderncvcolor{orange}                               % color options 'black', 'blue' (default), 'burgundy', 'green', 'grey', 'orange', 'purple' and 'red'
%\renewcommand{\familydefault}{\sfdefault}         % to set the default font; use '\sfdefault' for the default sans serif font, '\rmdefault' for the default roman one, or any tex font name
%\nopagenumbers{}     
% \renewcommand{\recomputelengths}

% adjust the page margins
\usepackage[scale=0.75]{geometry}
% \recomputelengths
%\setlength{\hintscolumnwidth}{3cm}                % if you want to change the width of the column with the dates
%\setlength{\makecvheadnamewidth}{10cm}            % for the 'classic' style, if you want to force the width allocated to your name and avoid line breaks. be careful though, the length is normally calculated to avoid any overlap with your personal info; use this at your own typographical risks...

    
% personal data
\name{Steven}{Kordonowy}
\title{Curriculum Vitae}
\email{skordono@ucsc.edu}
\homepage{stevenkordonowy.github.io}
\address{Santa Cruz, California}{}{}
\begin{document}
\makecvtitle
% \make
\section{Education}
\cventry{2022--present}{Doctor of Philosophy}{University of California - Santa Cruz}{CA}{Computer Science}{Advised by Dr. Alex Kolla\\Quantum computing, theoretical computer science}
\cventry{2019--2022}{Masters of Science}{University of Colorado - Boulder}{CO}{Computer Science}{Coursework in Algorithms and Complexity, Quantum Physics and Information\\Quantum science courses across Computer Science, Physics, Chemistry, and Engineering}
\cventry{2010--2014}{Bachelor of Science}{University of Denver}{CO}{Mathematics}{Minors: Computer Science, Physics, Psychology}
\cventry{2012}{Study Abroad}{Universidad de Buenos Aires}{Argentina}{}{}

\section{Papers}
\cventry{Sept 2023}{Approximation Algorithms for Quantum Max-d-Cut}{Collaborators: Charlie Carlson, Zack Jorquera, Alex Kolla}{\href{https://arxiv.org/abs/2309.10957}{preprint}}{}{}
\cventry{April 2023}{A quantum advantage over classical for local max cut}{Collaborators: Charlie Carlson, Zack Jorquera, Alex Kolla}{\href{https://arxiv.org/abs/2304.08420}{preprint}}{}{}

\section{Presentations and Posters}
\cventry{Jan 2023}{Poster: Approximation Algorithms for Quantum Max-d-Cut}{}{QIP 2023}{}{}
\cventry{Jan 2023}{Poster: A quantum advantage over classical for local max cut}{}{QIP 2023}{}{}

\section{Research Projects}
\cventry{2020--2023}{Quantum vs Classical Local Algorithms for Local Maxcut}{Collaborators:  Charles Carlson, Zack Jorquera, Alex Kolla}{}{}{}
\cventry{2023--present}{Optimizing quantum max-k-cut}{Collaborators:  Charles Carlson, Zack Jorquera, Alex Kolla}{}{}{}
\cventry{2012}{Summer Institute in Biostatistics}{Washington University}{St. Louis, MO}{}{}

\section{Teaching Experience}
\cventry{2022}{Quantum Computing (Instructor)}{CU-Boulder}{}{}{}
\cventry{2020, 2022}{Discrete Structures (Instructor)}{CU-Boulder}{}{}{}
\cventry{2021 - 2023}{Undergraduate Algorithms (TA)}{UCSC, CU-Boulder}{}{}{}
\cventry{2020 - 2023}{Quantum Computing (TA)}{UCSC, CU-Boulder}{}{}{}
\cventry{2020}{Linear Programming (TA)}{CU-Boulder}{}{}{}
\cventry{2019}{Computer Systems (TA)}{CU-Boulder}{}{}{}

\section{Awards}
\cventry{2023}{Outstanding TA Award for the Department of Computer Science and Engineering}{University of California Santa Cruz}{}{}{}
\cventry{2014}{Herbert J. Greenberg Award for Outstanding Achievements in Mathematics}{University of Denver}{}{}{}
\cventry{2013}{Outstanding Mathematics Junior}{University of Denver}{}{}{}
\cventry{2012}{Outstanding Mathematics Sophomore}{University of Denver}{}{}{}

\section{Professional Experience}
\cventry{2016--2019}{Software Engineer}{Nasdaq, Inc}{Lakewood, CO}{}{}
\cventry{2014--2016}{Software Engineer}{IntelliData, Inc.}{Greenwood Village, CO}{}{}

\section{Volunteer}
\cventry{2015--2018}{Tech Wizards}{4H}{Sun Valley Youth Center, Denver, CO
}{}{}

\section{Skills and Technologies}
Comfortable programming in any langue with expertise in Java, Python, and JavaScript

Quantum circuit programming experience with qiskit and qasm

CP/IP Networking

Common software engineering practices such as git, docker, and command line tools

Cloud computing technologies such as Kubernetes and Kafka 

Unix and Windows

\end{document}